\newenvironment{Abstract}%
    {\cleardoublepage\thispagestyle{plain}\null\vfill\begin{center}%
    \bfseries Abstract\end{center}}%
    {\vfill\null}
%\pagebreak
\cleardoublepage
\pagenumbering{roman}
\setcounter{page}{1}
\addcontentsline{toc}{chapter}{Abstract}
\begin{abstract}
\setlength{\parindent}{1cm}
The emergence of Cloud Computing and Virtualization gives new opportunities and solutions in terms of flexibility and services abstraction. In networking, the abstraction of services has been mainly shown through improving the networks programmability (SDN) and decreasing the dependence between networking functions (DNS, NAT,...) and the specific underlying hardware (NFV). Service Function Chaining (SFC) is a main breakthrough of services abstraction in networking. It presents networking services as an ordered set of basic networking functions, hence the increased flexibility. Nevertheless, such architecture would require improving the security aspect in order to ensure the networking functions are working as expected. In this work, we look into Blockchain, the technology laying behind the famous digital currency Bitcoin, and its potential impact on improving SFC security. Firstly, we define the Service Function Chaining and explain the different concepts related to it, then we analyse the various security aspects that can be improved in these networking architectures. As we consider mainly a blockchain-based solution for SFC security, we benchmark many blockchain solutions for this use case, namely, but not exclusively, Namecoin, Blockstack and Hyperledger, which allows us, eventually, to choose Bitcoin's Blockchain as the most convenient solution for our problem space.    \\[0.5cm]
\textbf{Keywords:} SFC, NFV, Blockchain, Service Functions, Accountability.
\end{abstract} 


